%%
%% The abstract is a short summary of the work to be presented in the
%% article.
\begin{abstract}
	The following paper deals with the question if the technology of a decentralized web could possibly be used in the future while replacing the currently well-established client-server architecture.
	
	To discuss this question properly a wide range of knowledge over the current situation and the functionality of a decentralized web is necessary.
	
	The paper starts with the explanation, why the term Web3 is used over the counterpart Web 3.0, which is because Web 3.0 references more the concept of a semantic web, while Web3 is mostly used while talking about the use case of blockchain in a new era of the internet.
	
	It continues by giving insights on how the client-server architecture works and how the technology evolved over time to it's current state.
	
	Also the major problems this architecture involves are extensively described, because these are the main reasons why a development towards a decentralized web is strongly needed. This includes (1) huge privacy issues regarding tech giants like GAFA
	\footnote{
		\textbf{G}oogle, \textbf{A}pple, \textbf{F}acebook und \textbf{A}mazon
	}
	stockpiling customer data (2) the need of web services or platforms which act as intermediaries. Not only trusting them is a huge issue, these servers are more likely to be attacked which sometimes ends in downtimes or even data thefts.
	
	\smallskip
	
	These problems could be solved by peer-to-peer networks and the use of blockchain. This massively increasing technology enables clients to not only use web services but provide them to others simultaneously. In this case the omission of intermediate platforms holds huge advantages, amongst others (1) security improvements, because there is no such thing like one server that could be attacked (2) control over private data, because everyone is in the position to delete his data whenever he wants.
	
	The problem with this technology lies in the implementation, because bugs which are not located beforehand can end in huge problems.
	
	\smallskip
	
	Therefore it is still a long way to a decentralized web, but there are already many organizations and companies working on it to improve every single aspect, in order to fulfill a transition to a free and safe web as it was intended to by it's founder.	
	
\end{abstract}

