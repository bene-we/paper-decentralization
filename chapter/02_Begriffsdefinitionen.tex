\section{Begriffsabgrenzung}

Vor dem tieferen Einstieg in die Materie werden an dieser Stelle zuerst die wichtigen Begrifflichkeiten geklärt, die zum Verständnis der folgenden Ausarbeitung nötig sind.

Es geht dabei um die Begriffe Web1, Web2 und Web3 im Gegensatz zu Web~1.0, Web~2.0 und Web 3.0. 
Während den Begriffen Web1 und Web~1.0 beziehungsweise Web2 und Web~2.0 dieselbe Bedeutung zukommt, gibt es bei Web3 und Web 3.0 signifikante Unterschiede. 
Denn der Begriff Web 3.0 wird in den meisten Fällen mit dem {\itshape semantischen Web} in Verbindung gesetzt. 
Dieses Konzept wurde, wie das WWW selbst, von Tim Berners-Lee bereits 1994 ins Leben gerufen und im Jahr 2001 in einem Artikel veröffentlicht (Vgl. \cite{Sweeney.2016}).
Diese Art des Internets beschreibt dabei die Erweiterung des Web~2.0 dahingehend, den Daten im Web eine Bedeutung zu verleihen und diese Bedeutung Maschinen durch Metadaten zugänglich zu machen. Das Ziel ist ein "klügeres" Internet, dass die Fragen von Nutzern auf menschlicher Ebene beantworten kann. Suchmaschinen im semantischen Web sollen beispielsweise unterscheiden können, ob ein Nutzer gerade ein Geldinstitut oder eine Parkbank sucht, wenn der Begriff "Bank" eingegeben wird (Vgl. \cite{Hilz.}). 
Für einen Einstieg in diesen Themenbereich wird die von Kate Ray im Jahr 2010 veröffentlichte Dokumentation
\footnote{
	\url{https://vimeo.com/11529540}
}
empfohlen, in der unter anderem Herr Berners-Lee interviewt wird. 
Diese Arbeit soll sich jedoch nicht um die Thematik des Verknüpfens von Daten handeln, sondern beschäftigt sich mit dem Einsatz von Dezentralisierung und Blockchain in der nächsten Generation des Webs. 
Um diesen Unterschied zu signalisieren, wird der Term Web3 konsistent verwendet, wie es auch Frau Voshmgir im Buch Token Economy zu ebenjenem Thema praktiziert. Eine Suche nach dem Term ``web3`` liefert zusätzlich die Library \verb|web3.js| zurück, die eine Anbindung an ein Ethereum Node, also eine Instanz eines Ethereum-Blockchain-Netzes bietet (Vgl. \cite{Ethereum.2019}). Auch die sogenannte Web3 Foundation (Vgl. \cite{Web3Foundation.2019}) beschäftigt sich mit dem Einsatz von Blockchain in der nächsten Generation des Internets, und bei den Web3 Summits geht es ebenso um Dezentralisierung (Vgl. \cite{Web3Summit.2019}). 
Die Schreibweisen Web1 und Web2 sind weniger bekannt als Web~1.0 und Web~2.0, und werden aus diesem Grund nicht verwendet.

